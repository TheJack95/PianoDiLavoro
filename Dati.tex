%----------------------------------------------------------------------------------------
%   USEFUL COMMANDS
%----------------------------------------------------------------------------------------

\newcommand{\dipartimento}{Dipartimento di Matematica "Tullio Levi-Civita"}

%----------------------------------------------------------------------------------------
% 	USER DATA
%----------------------------------------------------------------------------------------

% Data di approvazione del piano da parte del tutor interno; nel formato GG Mese AAAA
% compilare inserendo al posto di GG 2 cifre per il giorno, e al posto di 
% AAAA 4 cifre per l'anno
\newcommand{\dataApprovazione}{Data}

% Dati dello Studente
\newcommand{\nomeStudente}{Giacomo}
\newcommand{\cognomeStudente}{Corrò}
\newcommand{\matricolaStudente}{1122451}
\newcommand{\emailStudente}{giacomo.corro@studenti.unipd.it}
\newcommand{\telStudente}{+ 39 340 83 36 735}

% Dati del Tutor Aziendale
\newcommand{\nomeTutorAziendale}{Andrea}
\newcommand{\cognomeTutorAziendale}{Alberti}
\newcommand{\emailTutorAziendale}{info@aweba.it}
\newcommand{\telTutorAziendale}{+ 39 392 86 67 283}
\newcommand{\ruoloTutorAziendale}{Amministratore Delegato}

% Dati dell'Azienda
\newcommand{\ragioneSocAzienda}{AwebA}
\newcommand{\indirizzoAzienda}{Via Tofane 9, 31100, Treviso (TV)}
\newcommand{\sitoAzienda}{http://aweba.it/}
\newcommand{\emailAzienda}{info@aweba.it}
\newcommand{\partitaIVAAzienda}{P.IVA 04422050262}

% Dati del Tutor Interno (Docente)
\newcommand{\titoloTutorInterno}{Prof.}
\newcommand{\nomeTutorInterno}{Paolo}
\newcommand{\cognomeTutorInterno}{Baldan}

\newcommand{\prospettoSettimanale}{
    % Personalizzare indicando in lista, i vari task settimana per settimana
    % sostituire a XX il totale ore della settimana
    \begin{itemize}
        \item \textbf{Prima Settimana (40 ore)}
        \begin{itemize}
            \item Presentazione da parte del cliente della sua realtà aziendale (per via telematica);
            \item Meeting telematici con soggetti coinvolti nel progetto per discutere i requisiti e le richieste relativamente al sistema da sviluppare;
            \item Presa visione dell’infrastruttura esistente;
            \item Formazione sulle tecnologie adottate;
        \end{itemize}
        \item \textbf{Seconda Settimana (40 ore)} 
        \begin{itemize}
            \item Formazione sulla gestione degli ordini e consegna;
            \item Analisi del problema e delle richieste del cliente;
        \end{itemize}
        \item \textbf{Terza Settimana (40 ore)} 
        \begin{itemize}
            \item Preparazione ambiente di lavoro;    
            \item Installazione CMS e componenti aggiuntivi;
        \end{itemize}
        \item \textbf{Quarta Settimana (40 ore)} 
        \begin{itemize}
            \item Progettazione del portale Web (interfaccia grafica e funzionalità);
            \item Visione bozza grafica da parte del cliente e approvazione;
            \item Sviluppo front-end (HTML, CSS, JavaScript);
        \end{itemize}
        \item \textbf{Quinta Settimana (40 ore)} 
        \begin{itemize}
            \item Sviluppo front-end (HTML, CSS, JavaScript);
            \item Integrazione componenti aggiuntivi (PHP);
        \end{itemize}
        \item \textbf{Sesta Settimana (40 ore)} 
        \begin{itemize}
            \item Implementazione funzionalità pagamenti (PHP);
            \item Implementazione funzionalità spedizione (PHP);
        \end{itemize}
        \item \textbf{Settima Settimana (40 ore)} 
        \begin{itemize}
            \item Test pagamenti;
            \item Test funzionalità di spedizione;
            \item Eventuale correzione errori;
        \end{itemize}
        \item \textbf{Ottava Settimana (20 ore)} 
        \begin{itemize}
            \item Stesura documentazione relativa ad analisi e progettazione;
            \item Messa online dell'ecommerce;
            \item Presentazione portale Web al cliente;
            \item Analisi e stima di nuove richieste dei cliente;
        \end{itemize}
    \end{itemize}
}

% Indicare il totale complessivo (deve essere compreso tra le 300 e le 320 ore)
\newcommand{\totaleOre}{304}

\newcommand{\obiettiviObbligatori}{
	\item \underline{\textit{O01}}: Realizzazione software;
	\item \underline{\textit{O02}}: Realizzazione front-end (HTML e CSS);
	\item \underline{\textit{O03}}: Realizzazione back-end (PHP);
	\item \underline{\textit{O04}}: Messa online del sito;
}

\newcommand{\obiettiviDesiderabili}{
	\item \underline{\textit{D01}}: analisi e stima nuove richieste clienti;
	\item \underline{\textit{D02}}: progettazioni eventuali nuove funzionalità;
}

\newcommand{\obiettiviFacoltativi}{
	\item \underline{\textit{F03}}: implementazione di ulteriori metodi di pagamento;
}