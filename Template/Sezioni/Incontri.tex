\section*{Incontri}
Nonostante le disposizioni del governo vigenti in data 9 aprile 2020 in materia di gestione dell'emergenza epidemiologica da COVID-19 siano valide solo fino al 14 aprile 2020, tutte le interazione studente-cliente e studente-tutor saranno esclusivamente per via telematica in vista di un'ulteriore proroga delle attuali disposizioni. La modalità di lavoro sarà smart-working. \\
Nel caso di riapertura degli uffici, lo studente potrà recarsi in sede e avere contatti diretti con cliente e tutor.

\subsection*{Interazione tra studente e tutor aziendale}
Regolarmente, (almeno una volta la settimana) ci saranno incontri diretti con il tutor aziendale \nomeTutorAziendale\ \cognomeTutorAziendale\ per verificare lo stato di avanzamento, chiarire eventualmente gli obiettivi, affinare la ricerca e aggiornare il piano stesso di lavoro. Questi incontri si svolgeranno di persona, se possibile, oppure per via telematica.

\subsection*{Interazione tra studente e cliente}
Nelle prime fasi dello svolgimento del progetto, lo studente dovrà interagire con il cliente per capire le sue esigenze e trovare le soluzioni che meglio si adattano alla sua realtà aziendale. Sarà compito dello stagista presentare al cliente i vari avanzamenti del progetto, in modo da avere feedback sulla corretta realizzazione dal punto di vista delle funzionalità, della presentazione grafica e della gestione degli ordini effettuati attraverso l'e-commerce. \\
Eventuali modifiche o ulteriori richieste saranno comunicate dal cliente allo stagista, che dovrà poi analizzarle, presentale al tutor e comunicare al cliente le decisioni prese. \\
Le interazione tra studente e cliente saranno per via telematica (video-conferenze o chiamate telefoniche).